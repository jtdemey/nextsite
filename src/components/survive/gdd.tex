\documentclass[11pt]{article}

\usepackage{textcomp}

\title{The Roots Run Deep in Caledonia Forest}
\author{A game by John Torsten}

\begin{document}
	\maketitle
	\tableofcontents
	\clearpage

	\section{Terms}
	
	\begin{itemize}
		\item \textbf{TRRDCF}: abbreviation for \textit{The Roots Run Deep in Caledonia Forest}
		\item \textbf{Locale}: A distinct physical location in the game world, such as a room or a clearing in the woods.
		\item \textbf{Region}: An area in the game world with distinct environmental characteristics consisting of several locales.
	\end{itemize}
	
	
	\section{Concept}
	
	\textit{The Roots Run Deep in Caledonia Forest} (abbr. \textit{TRRDCF}) is a text-based game about finding a way to escape an unbelievably hostile forest tucked in the desolate corners of Caledonia County, Vermont in the year of 1987.
	
	\textit{TRRDCF}'s core gameplay revolves around exploring the world, interacting with the environment, and fighting enemies by typing textual commands into a console. Unlike most classic text-based adventure games, however, \textit{TRRDCF} runs in real-time and does not wait for the player's input.
	
	\section{Tone}
	
	\textit{TRRDCF}'s tone is gothic, macabre, and unsettling. The forest hosts the remnants of an uncanny civilization that has long since fallen to supernatural disarray, its existence hidden from locals due to the steep mountains and treacherous terrain that characterizes the environment. The remaining living inhabitants of the area are violent creatures of paranormal and mythical demeanor. Nothing is forgiving, and many things are lethal.
	
	\section{Mechanics}
	
	\subsection{Exploration}
	
	The majority of gameplay takes place in \textbf{Exploration mode}. In this mode, the player is able to traverse the map, examine paths and features, interact with things, and regulate their vital metrics.
	
	\textbf{Exploration mode} is split into four distinct views discussed later in the UI section.
	
	\begin{itemize}
		\item The \textbf{Console View} contains the command console with which the player enters commands and a scrollable section with the textual output of the game.
		\item The \textbf{Inventory View} is designed for interacting with items in the player's inventory and the environment.
		\item The \textbf{Map View} provides an overhead map to ease the player's navigation through the treacherous woods.
		\item The \textbf{Information View} provides a consolidated view of important information about the player's state, including vital statistics and any active modifiers or induced effects.
	\end{itemize}

	\textbf{Using Commands}
	\newline
	
	To perform actions, the player types commands into the console, such as "go north", which would cause the player to attempt to travel to the locale directly to the north. The resulting output of an action is printed to the console pane in the \textbf{Console View}.
	
	Most commands have several aliases; typing "go n" or even just "north" would accomplish the same action in this case.
	
	All player actions can be performed in the \textbf{Console View} by typing commands into the console, however the three other views provide alternative interfaces that ease interaction with items and the environment. Commands are discussed more thoroughly in the \textbf{Commands} section.\linebreak
	
	\textbf{Navigation}
	\newline
	
	The game's map can be split into several \textbf{regions}, which are each a collection of connected \textbf{locales}. Locales can contain exits in twelve possible directions: the eight cardinal directions, up, down, inside, and outside.
	
	The player can use the EXAMINE command to receive textual output regarding which exits are present in their current locale. The player can use the GO command to attempt to navigate in a direction.
	
	\subsection{Vitality}
	
	The player has four vital metrics that they must take action to regulate or die. These metrics are:
	
	\begin{itemize}
		\item \textit{Health}: A measure of physical vitality. 100 represents being in ideal physical state while 0 represents death.
		\item \textit{Sanity}: A measure of mental stature. 100 represents a calm, sane mental state while 0 represents a neurotic insanity -- a complete inability to distinguish hallucinations from reality.
		\item \textit{Energy}: A measure of physical energy. 100 represents being energetic while 0 represents complete exhaustion and involuntary immobility.
		\item \textit{Temperature}: A measure of body temperature. Unlike the previous metrics which are quantified on a scale of 0 to 100, this is a standard temperature readout in Fahrenheit and Celsius representing the main character's internal body temperature. 37\textdegree C (98.6\textdegree F) is considered an ideal body temperature. Dropping below this temperature results in freezing while exceeding this temperature results in overheating. Both yield negative effects, eventually resulting in death.
	\end{itemize}

	These vital metrics can be regulated through the use of items
	
	\subsection{Combat}
	
	When the player enters a locale occupied by an enemy or an enemy enters the locale occupied by the player, the player exits \textbf{Exploration mode} and enters \textbf{Combat mode}. The player will stay in this mode until the enemy is subdued or the player dies.
	
	\section{Commands}
	
	\section{Game Entities}
	\subsection{Items}
	
	\subsection{Enemies}
	
	\subsection{Bosses}
	
	\subsection{Weapons}
	
\end{document}